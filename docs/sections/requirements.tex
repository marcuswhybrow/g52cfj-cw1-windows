\section{Requirements}

\subsection{Draw an appropriate background}

\begin{itemize}
	\item The drawing of tiles is taken care of by the \verb!GameTileManager! class which is a subclass of \verb!TileManager!.
	\item The \verb!DrawTileAt! method has been overridden to draw 5 different types of tile. These are \emph{empty space}, \emph{vertical wall}, \emph{horizontal wall}, \emph{intersection wall} and \emph{hole}. The three wall tiles and the hole are drawn using the \verb!JPGImage! class from images created in Photoshop, whereas all empty space tiles are drawn as a grey rectangle.
	\item This class also contains new methods for testing whether an \verb!Actor! (base class for ball) will collide with a particular tile. Collision detection is done separately for each axis so you will see methods such as \verb!CheckCollisionVertical! and the private methods for checking collision for each of the different possible tiles (e.g. \verb!CollisionVerticalWallVertical!).
	\item The background changes due to player interaction when the player ball picks up a powerup. This is handled in the \verb!Player::CheckPowerups! method which is called at the very end of the \verb!Player::DoUpdate! method. The \verb!Player! gets the tile from the manager which it is currently on, and then evaluates whether there is a powerup in that tile, and whether or not the player ball is intersecting it. If it is the tile is replaces with an \emph{empty} tile and a method is called on the \verb!Player! to enable that power up.
	\item \verb!GameMain::SetupBackgroundBuffer! is overridden to draw a tiled background using the aforementioned \verb!GameTileManager! in the \verb!PLAYING! state and a grey background for the \verb!INTRO!, \verb!COMPLETED!, \verb!GAME_OVER! and \verb!HIGH_SCORES! states.
\end{itemize}

\subsection{Have moving objects}

\begin{itemize}
	\item This game has multiple moving objects, one is controlled by the player and the rest are controlled by the game (AI).
	\item The \verb!Actor! base class inherits from \verb!DisplayableObject! and overrides the the \verb!DoUpdate! method to change the position of each actor based upon its current velocity, after tacking into account any obstacles such as walls.
\end{itemize}

\subsection{Have interaction between the moving objects and the background}

\begin{itemize}
	\item \verb!Actor!s cannot pass through walls and fall down holes, this effect is upheld by collision logic contained with the (for example) \verb!GameTileManager::CheckCollisionHorizontal! method and checked on \verb!DoUpdate! in each \verb!Actor!.
	\item The \verb!Infected! (enemy ball) which inherits from \verb!Actor! is placed in a new level at a random position in its constructor, but only inside of tiles which are empty spaces. This is another level of (pre) interaction between movable objects and the background.
	\item When an actor is colliding with a hole background tile it puts the \verb!Actor! into a state where it animates the ball so it appears to fall down the hole. The movement of the ball is directed towards the the centre of the hole, and the ball stops moving when it reaches the centre.
\end{itemize}

\subsection{Provide user interaction by handling keyboard or mouse input}

\begin{itemize}
	\item The \verb!Player::DoUpdate! method gets the current x an y positions of the mouse, and calculates the angle necessary to point the player at the mouse, and then sets the player ball direction to be that angle.
	\item \verb!DoUpdate! also makes a call to \verb!Player::CheckKeys! which watches for the space key being released (as in pressed down and then released again) and then cycles the colour of the player ball to the next colour. This is what allows the player ball to change from red to green to blue and back to red again.
	\item \verb!GameMain::KeyUp! is overridden to provide some extra game wide functionality. The \verb!S! key when released and the game is in the \verb!INTRO! state starts the game at the first non-training level. When \verb!S! is pressed in the \verb!PLAYING! state the level is ended and the next level begins (skipping the level). And when \verb!H! is pressed in the \verb!INTRO! state the game changes to the \verb!HIGH_SCORES! state displaying the high scores list.
	\item \verb!GameMain::KeyDown! is overridden to provide actions more quickly (i.e. when the key is pressed down, not released). Pressing \verb!Esc! whilst in the INTRO state exits the game, whereas pressing it in any other state returns the game back to the \verb!INTRO! state.
\end{itemize}

\subsection{Provide AI-controlled objects}

\begin{itemize}
	\item \verb!Infected! objects are controlled completely by the game, the position of the \verb!Player! instance and the \verb!Player! instance's state.
	\item The \verb!Infected::DoUpdate! method calculates the angle at which it must point in order to face the player, and then its angle is updated to this angle every frame.
	\item The speed with which the \verb!Infected! object travels towards the \verb!Player! is determined in the \verb!DoUpdate! method by its proximity to the player. It increases in speed the closer it gets, creating a lunging animation whenever the player ball draws near. When a certain distance from the player an \verb!Infected! will not move, giving the impression of a certain area of perception.
	\item Each \verb!Infected! asks the \verb!Player! instance whether or not it should be chasing the player ball or running from it in the \verb!DoUpdate! method, this logic is contained with the a method called \verb!Player::ShouldAttract! and the returned boolean is determined by the current colour of the player ball, the colour of the \verb!Infected! calling the method and any powerups which the player ball might have active.
	\item The final thing the \verb!DoUpdate! method does is call the \verb!Infected::CheckIntersections! method, which iterates over all other actors and for each checks if this actor is colliding with that actor. If it is, and that actor is infectable as determined by a method called \verb!bool Actor::IsInfectable!, the slower \verb!Actor! assumes the colour of the faster one. This method also checks if this \verb!Infected! intersects with player, and asks the game to remove this \verb!Infected! if it does.
	\item If removed from play the \verb!Infected::HasBeenRemoved! method is called. The game is given points or penalised points for killing the \verb!Infected!.
	\item \verb!TutorialInfected! objects subclass \verb!Infected! and work in the exact same way, however in \verb!TutorialInfected::HasBeenRemoved! they progress the game to the next level, allowing for completion of training levels before every single \verb!Infected! has been killed.
	
\end{itemize}

\subsection{Load data from files}

\begin{itemize}
	\item The \verb!GameMain::LoadLevel! method loads in a file which is written in a syntax for defining level layouts. This level information is stored in a 3 dimensional array of characters for later use in the \verb!GameMain::SetupBackgroundBuffer! method.
	\item Each line in the level file is a row of tiles for the background of a level, and each character in that line matches a column in the background tile layout.
	\item Different characters are then assigned a numerical value for use in the tile manager, instructing different tiles to be drawn in different ways, and detect collisions in different ways as is appropriate for the visual representation of that tile.
	\item The syntax of a level layout file is as follows: A `\emph{space}' is an empty tile, the lowercase `\verb!x!' is a hole, a `\verb!|!' is a vertical wall, a `\verb!-!' is a horizontal wall, a `\verb!+!' is an intersection wall (connects on all four sides), `\verb!P!' is a tile with a power powerup, `\verb!Z!' is a tile with a speed powerup, `\verb!S!' is the starting location for the player and `\verb!C!' is visually represented as an empty space tile but is a tile where `\verb!Infected!' objects cannot spawn, i.e. a \verb!C!lear tile.
	\item When a level is loaded in \verb!LoadLevel! the player first player start position (\verb!S!) found in the file (read from top to bottom, left to right) is taken to be the starting point for the player. The player is started in the centre of that tile. The x and y value for each level is then stored in an array for use when setting up levels.
	\item All levels are loaded from file once in the \verb!GameMain::InitialiseObjects! method at the start of the game.
	\item Levels are stored in a folder named \verb!levels/!.
\end{itemize}

\subsection{Save and load information}

\begin{itemize}
	\item The 10 highest scores achieved in the game are saved in a file called \verb!highscrores.txt!.
	\item If this file does not exist, the game will create it when needed and fill it with 10 zeros.
	\item \verb!highscores.txt! is read in the \verb!GameMain::SetupHighScores! method, which is called whenever the high scores list is rendered, this includes the \verb!COMPLETED!, \verb!GAME_OVER! and \verb!HIGH_SCORES! states.
	\item When \verb!SetupHighScores! is called in the \verb!COMPLETED! or \verb!GAME_OVER! states, it is because the player has just completed all the levels or has died. In both cases the total amount of points needs to be entered into the high scores list if necessary.
	\item When entering a new score an array of the current scores in memory is iterated over from highest to lowest score. If the current score is found to be higher than a value in this array it replaces that value. The value which was at that position and all others after it are then shifted down to the next position. This maintains the ordering of scores from highest to lowest. The array of 10 highest scores whether altered or not, is now written back to the \verb!highscores.txt! file overriding its old value.
	\item If in the \verb!HIGH_SCORES! state the values are only read from file, and not altered in memory and no writing to the file takes place.
\end{itemize}

\subsection{Display status information on the screen}
\label{sec:status-info}

\begin{itemize}
	\item Whilst in the \verb!PLAYING! state, the \emph{level} which is in play and the amount of \emph{points} accumulated so far are displayed in the top left of the window.
	\item It may not be regarded as status information, but during the training levels textual explanations are drawn over the background but behind the \verb!Actor!s.
	\item All drawing of strings is handled in the \verb!GameMain::DrawStrings! method, which performs differently depending on different states and the current level if in the \verb!PLAYING! state.
	\item The status of the all actors, as in ``state" is displayed via the colour of each actor. The player then can intuit that with a green player, a green enemy ball would be in the attracting state, chasing the player down.
	\item The state of each ball with regard to attracting/repelling could be visually depicted via say a dot present in the centre of the actor if it is attracted to the player. However this information is intentionaly not displayed leaving this task to the human players brain, which makes the whole process fun.
	\item The state of the \verb!Player!'s enabled powerups is visually shown by transforming into a new colour. While the speed powerup is active (which makes all enemy balls repel, gives the player increased speed, and allows all colours to gain the player points on consumption) the player ball is made to be a white colour.
	\item When the speed powerup is less than one second away from being removed, the player ball is changed to a light grey colour, symbolising the fading of the powerup.
	\item When the speed power is actually removed, the player ball's colour is returned to the colour it was before picking up this powerup, showing a return to the previous state.
	\item When the power powerup is active (which attracts all enemy balls, and allows the player to gain points from consuming any colour) a black border is drawn around the player symbolising a toughened shell.
	\item When the power powerup is less than one second away from being removed, the border colour is changed to become a---slightly ligher than black---dark grey, symbolising the wearing off of the power powerup.
	\item When actually removed the power powerup's border border is removed.
\end{itemize}

\subsection{Support different states}

\begin{itemize}
	\item The game has 5 states: \verb!INTRO! displays a welcome screen, \verb!PLAYING! displays the current level and moves all actors etc., \verb!COMPLETED! shows the high scores list and your total final score as well as highlighting its position in the list if it made it in, \verb!GAME_OVER! does the same as \verb!COMPLETED! with a different message explaining that you died, and \verb!HIGH_SCORES! simply shows a list of the highest scores. All these states are defined in \verb!pulic const enum State! in \verb!GameMain.h!.
	\item The currently active state is stored in a private variable \verb!_state! in \verb!GameMain!.
	\item The default starting state is \verb!INTRO!.
	\item The current state is only ever changed via a call to the \verb!GameMain::ChangeState! method, which performs cleanup when exiting certain states, calls \verb!SetupBackgroundBuffer! to load in the correct background for the new current state, and then calls \verb!Redraw(true)! to redraw the entire screen.
	\item This approach makes it incredibly easy to move through states, for example hitting \verb!H! whilst in on the title screen just calls \verb!ChangeState(HIGH_SCORES)!, and pressing escape whilst viewing the high scores just calls \verb!ChangeState(INTRO)!.
	\item Aside from game wide state, there are other smaller states which exist on a per \verb!Actor! (ball) basis.
	\item Each \verb!Actor! has an \verb!isFallingInHole! state, which is true if it is should be animating itself down a hole.
	\item The player has a \verb!speedPowerupEnabled! and \verb!powerPowerupEnabled! state, which effect how the player is drawn, which has been explained in section \ref{sec:status-info}.
	\item The player also keeps track of a \verb!spaceIsDown! state for detecting when it moves from \verb!true! to \verb!false!, signifying the release of the space bar.
\end{itemize}