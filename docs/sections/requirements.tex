\section{Requirements}

\subsection{Draw an appropriate background}

\begin{itemize}
	\item The drawing of tiles is taken care of by the \verb!GameTileManager! class which is a subclass of \verb!TileManager!.
	\item The \verb!DrawTileAt! method has been overridden to draw 5 different types of tile. These are \emph{empty space}, \emph{vertical wall}, \emph{horizontal wall}, \emph{intersection wall} and \emph{hole}. The three wall tiles and the hole are drawn using the \verb!JPGImage! class from images created in Photoshop, whereas all empty space tiles are drawn as a grey rectangle.
	\item This class also contains new methods for testing whether an \verb!Actor! (base class for ball) will collide with a particular tile. Collision detection is done separately for each axis so you will see methods such as \verb!CheckCollisionVertical! and the private methods for checking collision for each of the different possible tiles (e.g. \verb!CollisionVerticalWallVertical!).
	\item The background changes due to player interaction when the player ball picks up a powerup. This is handled in the \verb!Player::CheckPowerups! method which is called at the very end of the \verb!Player::DoUpdate! method. The \verb!Player! gets the tile from the manager which it is currently on, and then evaluates whether there is a powerup in that tile, and whether or not the player ball is intersecting it. If it is the tile is replaces with an \emph{empty} tile and a method is called on the \verb!Player! to enable that power up.
	\item \verb!GameMain::SetupBackgroundBuffer! is overridden to draw a tiled background using the aforementioned \verb!GameTileManager! in the \verb!PLAYING! state and a grey background for the \verb!INTRO!, \verb!COMPLETED!, \verb!GAME_OVER! and \verb!HIGH_SCORES! states.
\end{itemize}

\subsection{Have moving objects}

\begin{itemize}
	\item This game has multiple moving objects, one is controlled by the player and the rest are controlled by the game (AI).
	\item The \verb!Actor! base class inherits from \verb!DisplayableObject! and overrides the the \verb!DoUpdate! method to change the position of each actor based upon its current velocity, after tacking into account any obstacles such as walls.
\end{itemize}

\subsection{Have interaction between the moving objects and the background}

\begin{itemize}
	\item \verb!Actor!s cannot pass through walls and fall down holes, this effect is upheld by collision logic contained with the (for example) \verb!GameTileManager::CheckCollisionHorizontal! method and checked on \verb!DoUpdate! in each \verb!Actor!.
	\item The \verb!Infected! (enemy ball) which inherits from \verb!Actor! is placed in a new level at a random position in its constructor, but only inside of tiles which are empty spaces. This is another level of (pre) interaction between movable objects and the background.
	\item When an actor is colliding with a hole background tile it puts the \verb!Actor! into a state where it animates the ball so it appears to fall down the hole. The movement of the ball is directed towards the the centre of the hole, and the ball stops moving when it reaches the centre.
\end{itemize}

\subsection{Provide user interaction by handling keyboard or mouse input}

\begin{itemize}
	\item
\end{itemize}

\subsection{Provide AI-controlled objects}

\begin{itemize}
	\item
\end{itemize}

\subsection{Load data from files}

\begin{itemize}
	\item
\end{itemize}

\subsection{Save and load information}

\begin{itemize}
	\item
\end{itemize}

\subsection{Display status information on the screen}

\begin{itemize}
	\item
\end{itemize}

\subsection{Support different states}

\begin{itemize}
	\item
\end{itemize}